\chapter{Geometry}
Geometry (from the Ancient Greek: geo- "earth", -metron "measurement") is a branch of mathematics concerned with questions of shape, size, relative position of figures, and the properties of space.


\section{Plane Geometry}
\subsection{Point, Lines}
Distance between two points $P=(x_1,y_1)$ and $Q=(x_2,y_2)$ in $\mathbb{R}^2$ : $ \overline{PQ} = \sqrt{(x_1-x_2)^2+(y_1-y_2)^2} $

Midpoint between P and Q: $\left(\frac{x_1+x_2}{2},\frac{y_1+y_2}{2}\right)$

Lines equation given one point P: $ y-y_1 = m(x-x_1) $

Line equation given two points P and Q: $ y-y_1 = \frac{y_2-y_1}{x_2-x_1}(x-x_1) $

Distance of one point P from the line r with equation $ax+by+c=0$ : $ d(U,r) = \frac{\left|ax_1+by_1+c\right|}{\sqrt{a^2+b^2}}$


\section{Trigonometry}
\textbf{Definition:} Trigonometry (from Greek trigonon, "triangle" and metron, "measure") is a branch of mathematics that studies relationships involving lengths and angles of triangles.

\subsection{Goniometric functions}
\textbf{Definition:} In mathematics, the goniometric functions (also called circular functions, angle functions or trigonometric functions) are functions of an angle. They relate the angles of a triangle to the lengths of its sides. 

For the goniometric ratios for a point $p$ on the unit circle holds:
\[
\sin(\phi)=y_p~~,~~\cos(\phi)=x_p~~,~~\tan(\phi)=\frac{\sin(\phi)}{\cos(\phi)}
\]

The reciprocal functions secant (sec), cosecant (csc), and cotangent (cot) are the reciprocals of the cosine, sine, and tangent:
\[
\sec(\phi)=\frac{1}{\sin{\phi}}~~,\csc(\phi)=\frac{1}{\cos{\phi}}~~,~~\cot(\phi)=\frac{1}{\tan(\phi)}=\frac{\cos(\phi)}{\sin(\phi)}
\]

\subsubsection{Inverse goniometric functions}
\begin{tabular}{  | c || c | c | c | c | c | c | }
	\hline
	Function   &  $\sin$  &  $\cos$  &  $\tan$  &  $\sec$ &  $\csc$ &  $\cot$ \\
	\hline
	Inverse   &  $\arcsin$  &  $\arccos$  &  $\arctan$  &  arcsec  &  arccsc  &  arccot \\
	\hline	
\end{tabular}


\subsubsection{Pythagorean trigonometric identity}
\[
\sin^2(\phi)+\cos^2(\phi)=1 
\]


\subsubsection{Identities in terms of their complements and supplements}
\[
\sin\left(\frac{\pi}{2}-\phi\right)=\cos(\phi) \ \ , \ \ \cos\left(\frac{\pi}{2}-\phi\right)=\sin(\phi)
\]
\[
\sin\left(\frac{\pi}{2}+\phi\right)=\cos(\phi) \ \ , \ \ \cos\left(\frac{\pi}{2}+\phi\right)=-\sin(\phi)
\]

\[
\sin(\pi-\phi)=\sin(\phi) \ \ , \ \ \cos(\pi-\phi)=-\cos(\phi)
\]
\[
\sin(\pi+\phi)=-\sin(\phi) \ \ , \ \ \cos(\pi+\phi)=-\cos(\phi)
\]

\[
\sin\left(\frac{3}{2}\pi-\phi\right)=-\cos(\phi) \ \ , \ \ \cos\left(\frac{3}{2}\pi-\phi\right)=-\sin(\phi)
\]
\[
\sin\left(\frac{3}{2}\pi+\phi\right)=-\cos(\phi) \ \ , \ \ \cos\left(\frac{3}{2}\pi+\phi\right)=\sin(\phi)
\]


\subsubsection{Identities for negative angles}
\[
\sin(-\phi) = -\sin(\phi) \ , \ \cos(-\phi) = \cos(\phi)
\]


\subsubsection{Periodicity of trig functions}
\[
\sin(\phi+2\pi)=\sin(\phi) \ , \ \cos(\phi+2\pi)=\cos(\phi) \ , \ \tan(\phi+\pi)=\tan(\phi)
\]


\subsubsection{Ptolemy's identities}
\[
\sin(a+b)=\sin(a)\cos(b)+\cos(a)\sin(b)
\]
\[
\cos(a+b)=\cos(a)\cos(b)-\sin(a)\sin(b)
\]
\[
\sin(a-b)=\sin(a)\cos(b)-\cos(a)\sin(b)
\]
\[
\cos(a-b)=\cos(a)\cos(b)+\sin(a)\sin(b)
\]

\[
\tan(\alpha+\beta)=\frac{\tan(\alpha) + \tan(\beta)}{1-\tan(\alpha)\tan(\beta)} \text{ where } \alpha, \beta, \alpha+\beta \ne \frac{\pi}{2} + k\pi, \ k \in \mathbb{Z}
\]
\[
\tan(\alpha-\beta)=\frac{\tan(\alpha) - \tan(\beta)}{1+\tan(\alpha)\tan(\beta)} \text{ where } \alpha, \beta, \alpha-\beta \ne \frac{\pi}{2} + k\pi, \ k \in \mathbb{Z}
\]

\subsubsection{Double angle formulas}
\[
\sin(2\alpha) = 2\sin(\alpha)\cos(\alpha)
\]
\[
\cos(2\alpha)=\cos^2(\alpha)-\sin^2(\alpha) = 2\cos^2(\alpha)-1 = 1 - 2\sin^2(\alpha)
\]
\[
\tan(2\alpha) = \frac{2\tan(\alpha)}{1-2\tan^2(\alpha)} \text{ where } \alpha \ne \frac{\pi}{4}+k\frac{\pi}{2} \ \wedge \ \alpha \ne \frac{\pi}{2}+k\pi, \ k \in \mathbb{Z}
\]

\subsubsection{Half angle formulas}
\[
\sin^2\left(\frac{\alpha}{2}\right) = \frac{1-\cos(\alpha)}{2}
\]
\[
\cos^2\left(\frac{\alpha}{2}\right) = \frac{1+\cos(\alpha)}{2}
\]
\[
\tan\left(\frac{\alpha}{2}\right) = \frac{\sin(\alpha)}{1+\cos(\alpha)}
\]

\subsubsection{Product-to-sum identities}
\[
2\sin(\alpha)\sin(\beta) = \cos(\alpha-\beta) - \cos(\alpha+\beta)
\]
\[
2\cos(\alpha)\cos(\beta) = \cos(\alpha-\beta) + \cos(\alpha+\beta)
\]
\[
2\sin(\alpha)\cos(\beta) = \sin(\alpha-\beta) + \sin(\alpha+\beta)
\]

\subsubsection{Sum-to-product identities}
\[
\sin(\alpha)+\sin(\beta) = 2\sin\left(\frac{\alpha+\beta}{2}\right)\cos\left(\frac{\alpha-\beta}{2}\right)
\]
\[
\sin(\alpha)-\sin(\beta) = 2\cos\left(\frac{\alpha+\beta}{2}\right)\sin\left(\frac{\alpha-\beta}{2}\right)
\]
\[
\cos(\alpha)+\cos(\beta) = 2\cos\left(\frac{\alpha+\beta}{2}\right)\cos\left(\frac{\alpha-\beta}{2}\right)
\]
\[
\cos(\alpha)-\cos(\beta) = -2\sin\left(\frac{\alpha+\beta}{2}\right)\sin\left(\frac{\alpha-\beta}{2}\right)
\]

\subsubsection{Tangent half-angle substitution}
\[
\sin(\alpha) = \frac{2t}{1+t^2} \text{ where } t=\tan\left(\frac{\alpha}{2}\right) \text{ and } \alpha \ne \pi + 2k\pi
\]
\[
\cos(\alpha) = \frac{1-t^2}{1+t^2} \text{ where } t=\tan\left(\frac{\alpha}{2}\right) \text{ and } \alpha \ne \pi + 2k\pi
\]
\[
\tan(\alpha) = \frac{2t}{1-t^2} \text{ where } t=\tan\left(\frac{\alpha}{2}\right) \text{ and } \alpha \ne \frac{\pi}{2} + k\pi \ \wedge \ \alpha \ne \pi + 2k\pi
\]

\subsection{Conversions of common angles}

\begin{tabular}{  | c | c || c | c | c | c | }
	\hline
	Degrees          & Radians              & Sin                               & Cos                                               & Tan                                    & Cot               \\
	\hline
    $0^{\circ}$      & 0                    & 0                                 &  1                                                & 0                                      & $\pm\infty$        \\
    \hline	
    $15^{\circ}$     & $\frac{\pi}{12}$     & $\frac{\sqrt{6}-\sqrt{2}}{4}$       &  $\frac{\sqrt{6}+\sqrt{2}}{4}$                      & $2-\sqrt{3}$                            & $2+\sqrt{3}$       \\
    \hline		
    $30^{\circ}$     & $\frac{\pi}{6}$     & $\frac{1}{2}$      &   $\frac{\sqrt{3}}{2} $     &  $\frac{\sqrt{3}}{3}$    &  $\sqrt{3}$  \\
    \hline		
    $45^{\circ}$     & $\frac{\pi}{4}$     & $\frac{\sqrt{2}}{2}$   &   $\frac{\sqrt{2}}{2}$    &  1    &  1  \\
    \hline		
    $60^{\circ}$     & $\frac{\pi}{3}$     & $\frac{\sqrt{3}}{2}$   &   $\frac{1}{2}$    &  $\sqrt{3}$    &  $\frac{\sqrt{3}}{3}$  \\
    \hline		
    $75^{\circ}$     & $\frac{5}{12}\pi$     & $\frac{\sqrt{6}+\sqrt{2}}{4}$   &   $\frac{\sqrt{6}-\sqrt{2}}{4}$    &  $2+\sqrt{3}$    &  $2-\sqrt{3}$  \\
    \hline
    $90^{\circ}$     & $\frac{\pi}{2}$     & 1   &   0    &  $\pm\infty$   &  0  \\
    \hline
   	$105^{\circ}$     & $\frac{7}{12}\pi$     & $\frac{\sqrt{6}+\sqrt{2}}{4}$   &   $\frac{\sqrt{2}-\sqrt{6}}{4}$    &  $-2-\sqrt{3}$    &  $\sqrt{3}-2$  \\
    \hline
    $120^{\circ}$     & $\frac{2}{3}\pi$     & $\frac{\sqrt{3}}{2}$      &   $-\frac{1}{2} $     &  $-\sqrt{3}$    &  $-\frac{\sqrt{3}}{3}$  \\
    \hline	
  	$135^{\circ}$     & $\frac{3}{4}\pi$     & $\frac{\sqrt{2}}{2}$   &   $-\frac{\sqrt{2}}{2}$    &  -1    &  -1  \\
  	\hline
  	$150^{\circ}$     & $\frac{5}{6}\pi$     & $\frac{1}{2}$      &   $-\frac{\sqrt{3}}{2}$     &  $-\frac{\sqrt{3}}{3}$    &  $-\sqrt{3}$  \\
  	\hline	
    $165^{\circ}$     & $\frac{11}{12}\pi$   & $\frac{\sqrt{6}-\sqrt{2}}{4}$       &  -$\frac{\sqrt{6}+\sqrt{2}}{4}$     & $\sqrt{3}-2$   & $-\sqrt{3}-2$       \\
    \hline
    $180^{\circ}$     & $\pi$     & 0   &   -1   &  0    &  $\pm\infty$  \\
    \hline
	$195^{\circ}$     & $\frac{13}{12}\pi$   & $\frac{\sqrt{2}-\sqrt{6}}{4}$       &  -$\frac{\sqrt{6}+\sqrt{2}}{4}$     & $2-\sqrt{3}$   & $2+\sqrt{3}$       \\
	\hline
	$210^{\circ}$     & $\frac{7}{6}\pi$     & $-\frac{1}{2}$      &   $-\frac{\sqrt{3}}{2}$     &  $\frac{\sqrt{3}}{3}$    &  $\sqrt{3}$  \\
	\hline
  	$225^{\circ}$     & $\frac{5}{4}\pi$     & $-\frac{\sqrt{2}}{2}$      &   $-\frac{\sqrt{2}}{2}$     &  1    &  1  \\
  	\hline	
  	$240^{\circ}$     & $\frac{4}{3}\pi$     & $-\frac{\sqrt{3}}{2}$      &   $-\frac{1}{2}$     &  $\sqrt{3}$    &  $\frac{\sqrt{3}}{3}$  \\
  	\hline
     $255^{\circ}$     & $\frac{17}{12}\pi$   & $-\frac{\sqrt{6}+\sqrt{2}}{4}$       &  $\frac{\sqrt{2}-\sqrt{6}}{4}$     & $2+\sqrt{3}$   & $2-\sqrt{3}$       \\
     \hline
     $270^{\circ}$     & $\frac{3}{2}\pi$     & -1      &   0    &  $\pm\infty$    &  0  \\
     \hline	
    $285^{\circ}$     & $\frac{19}{12}\pi$   & $-\frac{\sqrt{6}+\sqrt{2}}{4}$       &  $\frac{\sqrt{6}-\sqrt{2}}{4}$     & $-2-\sqrt{3}$   & $\sqrt{3}-2$       \\
    \hline
    $300^{\circ}$     & $\frac{5}{3}\pi$     & $-\frac{\sqrt{3}}{2}$      &   $\frac{1}{2}$     &  $-\sqrt{3}$    &  $-\frac{\sqrt{3}}{3}$  \\
    \hline	
    $315^{\circ}$     & $\frac{7}{4}\pi$     & $-\frac{\sqrt{2}}{2}$      &   $\frac{\sqrt{2}}{2}$     &  -1   &  -1  \\
    \hline	
    $330^{\circ}$     & $\frac{11}{6}\pi$     & $-\frac{1}{2}$      &   $\frac{\sqrt{3}}{2}$     &  -$\frac{\sqrt{3}}{3}$   &  $-\sqrt{3}$  \\
    \hline	
     $345^{\circ}$     & $\frac{23}{12}\pi$     & $\frac{\sqrt{2}-\sqrt{6}}{4}$       &  $\frac{\sqrt{2}+\sqrt{6}}{4}$    &  $\sqrt{3}-2$    &  $-2-\sqrt{3}$  \\
     \hline	
     $360^{\circ}$     & $2\pi$   & 0       &  1     & 0   & $\pm\infty$       \\
     \hline
\end{tabular}
