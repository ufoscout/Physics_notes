\chapter{Calculus}
\textbf{Definition:} Calculus (from Latin calculus, literally 'small pebble', used for counting and calculations, as on an abacus) is the mathematical study of continuous change. It has two major branches:
\begin{itemize}  
	\item \textbf{differential calculus:} concerning rates of change and slopes of curves;
	\item \textbf{integral calculus:} concerning accumulation of quantities and the areas under and between curves
\end{itemize}
These two branches are related to each other by the fundamental theorem of calculus. 


\section{Limits}
\[
\lim_{x\rightarrow\pm\infty}P(x)=\lim_{x\rightarrow\pm\infty}a_nx^x=\pm\infty
\]
\[
\lim_{x\rightarrow\pm\infty}\frac{P(x)}{Q(x)}=\lim_{x\rightarrow\pm\infty}\frac{a_nx^x}{b_mx^m}=\lim_{x\rightarrow\pm\infty}\frac{a_n}{b_m}x^{n-m}=
\begin{cases}
	\infty & \text{if } n>m \\
	\frac{a_n}{b_m} & \text{if } n=m \\
	0 & \text{if } n<m \\
\end{cases}
\]

\[
\lim_{x\rightarrow0}\frac{\sin(x)}{x}=1 ~~,~~ \lim_{x\rightarrow0}\frac{1-\cos(x)}{x^2}=\frac{1}{2}
\]

\[
\lim_{x\rightarrow0}\frac{\tan(x)}{x}=1 ~~,~~ \lim_{x\rightarrow0}\frac{\arcsin(x)}{x}=1
\]

\[
\lim_{x\rightarrow0}\frac{\log_a(x+1)}{x}=\frac{1}{\ln(a)} ~~ (a>0) ~~,~~
\lim_{x\rightarrow0}\frac{\ln(x+a)}{x}=a
\]

\[
\lim_{x\rightarrow+\infty}x^\alpha=+\infty ~~,~~ \lim_{x\rightarrow0^+}x^\alpha=0 ~~~~ \alpha>0
\]
\[
\lim_{x\rightarrow+\infty}x^\alpha=0 ~~,~~ \lim_{x\rightarrow0^+}x^\alpha=+\infty ~~~~ \alpha<0
\]
\[
\lim_{x\rightarrow0}\frac{a^x-1}{x}=ln(a) ~~,~~ \lim_{x\rightarrow0}\frac{e^x-1}{x}=1
\]

\[
\lim_{x\rightarrow+\infty}a^x=+\infty ~~,~~ \lim_{x\rightarrow-\infty}a^x=0 ~~~~ a>1
\]
\[
\lim_{x\rightarrow+\infty}a^x=0 ~~,~~ \lim_{x\rightarrow-\infty}a^x=+\infty ~~~~ a<1
\]

\[
\lim_{x\rightarrow+\infty}\log_ax=+\infty ~~,~~ \lim_{x\rightarrow0^+}\log_ax=-\infty ~~~~ a>1
\]
\[
\lim_{x\rightarrow+\infty}\log_ax=-\infty ~~,~~ \lim_{x\rightarrow0^+}\log_ax=+\infty ~~~~ a<1
\]

\[
\lim_{x\rightarrow0}(1+x)^{1/x}={\rm e}~~,~~
\lim_{x\rightarrow\infty}\left(1+\frac{n}{x}\right)^x={\rm e}^n
\]
\[
\lim_{x\rightarrow\infty}\frac{x^p}{a^x}=0~~\mbox{als }|a|>1 ~~,~~
\lim_{x\rightarrow0}\frac{(1+x)^\alpha-1}{x}=\alpha ~~~ (a\in\mathbb{R})
\]
\[
\lim_{x\rightarrow0}\left(a^{1/x}-1\right)=\ln(a)~~,~~
\lim_{x\rightarrow\infty}\sqrt[x]{x}=1
\]

\subsubsection{Substitution rule}
\[
\text{given  } \lim_{x\rightarrow c}f(x)=l \text{   then   } \lim_{x\rightarrow c}g(f(x))=\lim_{y\rightarrow l}g(y) 
\]

\section{Differential calculus}
\textbf{Definition:} In mathematics, differential calculus is a subfield of calculus concerned with the study of the rates at which quantities change.





\section{Integral Calculus}
\textbf{Definition:} In mathematics, Integral calculus is a subfield of calculus in which the notion of an integral, its properties and methods of calculation are studied. It concerns accumulation of quantities and the areas under and between curves.





\section{Series}	
\textbf{Definition:} In mathematics, a series is, roughly speaking, a description of the operation of adding infinitely many quantities, one after the other, to a given starting quantity.



\begin{multicols}{3}
	
$ \sum_{k=1}^n k = \frac{n(n+1)}{2} $

$ \sum_{k=1}^n (2k-1) = n^2 $

$ \sum_{k=1}^n k^2 = \frac{n(n+1)(2n+1)}{6} $

$ \sum_{k=1}^n k^3 = \frac{n^2(n+1)^2}{4} $

$
\sum_{k=0}^{n}q^k = \begin{cases}
                         (n+1), & \text{for } q=1\\
                         \frac{1-q^{(n+1)}}{1-q}, & \text{for } q\neq1
                    \end{cases}
$
	
\end{multicols}
