\chapter{Arithmetic}
\textbf{Definition:} Arithmetic (from the Greek arithmos, "number") is a branch of mathematics that consists of the study of numbers, especially the properties of the traditional operations on them.

\section{Numbers}

$\mathbb{N} = \{0,1,2,3,...\} $ = natural numbers \\
$\mathbb{N^+} = \mathbb{N}\setminus\{0\}$ \\
$\mathbb{Z} = \{...,-3,-2,-1,0,1,2,3,...\} $ = integers \\
$\mathbb{Q}$ = rational numbers \\
$\mathbb{R}$ = real numbers \\
$\mathbb{C}$ = complex numbers

\section{Absolute Value}

Absolute value function: $ \left|x\right| = max\{x,-x\} = \sqrt{x^2} = \begin{cases}
x & \text{if } x\geq0\\
-x & \text{if } x<0
\end{cases} $

\section{Factorial}

Factorial: $ n! = \prod_{k=1}^{n}k  = \begin{cases}
0! &= 1\\
n! &= n(n-1)! \ \ \forall n\ge1
\end{cases} $

\section{Exponential}
Exponentiation is a mathematical operation, written as $b^n$, involving two numbers, the base b and the exponent n. When $n>0$ exponentiation corresponds to: $$b^n=\underbrace{b \times ... \times b}_\text{n times}$$

Properties:
\[
a^0=1 \ \ , \ \ a^{x+y}=a^xa^y \ \ , \ \ a^{x-y}=\frac{a^x}{a^y} \ \ , \ \ (a^x)^y=a^{xy} \ \ , \ \ (ab)^n = a^nb^n \ \ , \ \ a^{\frac{x}{y}}=\sqrt[y]{a^x}
\]

\section{Logarithm}
{\bf Definition}: $a^x=y\Leftrightarrow\log_a(y)=x $. For logarithms with
base $e$ one writes $\ln(x)$.

{\bf Rules}:

\begin{tabular}{ l l }
$ \displaystyle \log_a(1)=0 $ &
$ \displaystyle a^{log_a(y)}=y $ \\
$ \displaystyle \log_a(a^x)=x $ &
$ \displaystyle \log_a(x^y)=y\log_a(x) \ \forall x>0 \ \forall y \in \mathbb{R} $ \\
$ \displaystyle \log_a(x)=\frac{1}{\log_x(a)} $ &
$ \displaystyle \log_a\left(1/x\right)=-\log_a(x) $ \\
$ \displaystyle \log_a(xy) = \log_a(x)+\log_a(y) \  \forall x,y>0 $ &
$ \displaystyle \log_a(x/y) = \log_a(x)-\log_a(y) \ \forall x,y>0 $
\end{tabular}


